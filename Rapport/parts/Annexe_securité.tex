La sécurité est un sujet primordiale dans tous les laboratoires et tout travail comportant des risques se doit d'être encadré par des règles strictes. Notre activité expérimentale comportait différents manipulation à risque que nous avons identifiées:
\begin{itemize}
  \item Manipulation de l'azote liquide.
  \item Manipulation de la plaque chauffante.
  \item Manipulation d'instruments à haute tension.
  \item Manipulation de la perçeuse.
\end{itemize}

Pour chacun de ces aspects nous avons pris les précautions nécessaires pour minimiser les risques d'accidents. 
L'azote liquide est utilisée dans divers domaines de l'industrie et de la recherche. Nous l'avons utilisé pour pouvoir refroidir notre échantillon à très basse température. En effet la température de l’azote liquide (point d’ébullition à la pression atmosphérique) est de -196°C. 
Il existe divers dangers potentiels qui proviennent de la manipulation de l'azote liquide. Il existe quatre catégories principales de dangers reliés a l'azote liquide. Nous allons les présenter individuellements suivies des mesures de sécurité à respecter.

\paragraph*{Très basse température}
\begin{itemize}
  \item Lorsque le liquide cryogénique entre en contact avec la peau, il peut provoquer des 
brûlures par le froid appelées gélures. Des gélures sur une grande surface de la peau peuvent être mortelles. 
  \item Les très basses températures diminuent la résilience et la ductilité de certains 
matériaux, qui se fragilisent et peuvent se briser facilement. Ceci peut se traduire par des coupures ou par un renversement d'un produit dangereux.
  \item Lors de la manipulation de l'azote liquide l’air peut se condenser. Le condensat riche en oxygène augmente le risque d'incendie Dans le condensat qui 
s’égoutte, l’oxygène s’enrichit. Si celui-ci pénètre dans un matériau solide inflammable 
(ex. bois ou matériau d’isolation organique), il s’en suit un risque élevé d’incendie.
