La sécurité est un sujet primordial dans tous les laboratoires et tout travail comportant des risques se doit d'être encadré par des règles strictes. Notre activité expérimentale comportait différentes manipulation à risque que nous avons identifiées:
\begin{itemize}
  \item Manipulation de l'azote liquide.
  \item Manipulation de la plaque chauffante.
  \item Manipulation d'instruments à haute tension.
  \item Manipulation de la perçeuse.
\end{itemize}

Pour chacun de ces aspects nous avons pris les précautions nécessaires pour minimiser les risques d'accidents. 
L'azote liquide est utilisé dans divers domaines de l'industrie et de la recherche. Nous l'avons utilisé pour pouvoir refroidir notre échantillon à très basse température. En effet la température de l’azote liquide (point d’ébullition à la pression atmosphérique) est de -196°C. 
Il existe quatre catégories principales de dangers reliés à la manipulation de l'azote liquide. Nous allons les présenter individuellements suivies des mesures de sécurité à respecter.

\subsection*{Très basse température}

\paragraph*{Dangers}
\begin{itemize}
  \item Lorsque le liquide cryogénique entre en contact avec la peau, il peut provoquer des 
brûlures par le froid appelées gélures. Des gélures sur une grande surface de la peau peuvent être mortelles. 
  \item Les très basses températures diminuent la résilience et la ductilité de certains 
matériaux, qui se fragilisent et peuvent se briser facilement. Ceci peut se traduire par des coupures ou par un renversement d'un produit dangereux.
  \item Lors de la manipulation de l'azote liquide l’air peut se condenser. Le condensat riche en oxygène augmente le risque d'incendie 
\end{itemize}

\paragraph*{Mesures de sécurité}
\begin{itemize}
  \item Lors de toute manipulation de l'azote liquide il faut utiliser des gants approriés, des vêtements couvrant tout le corps, des lunettes de sécurité et des chaussures fermées.
  \item Le transport du liquide doit être réalisé dans des récipients adaptés, résistants aux basses températures et aux chocs. Lors du déplacement du récipient il faut éviter tout choc ou mouvement risquant le renversement du liquide. Le sol des pièces contenant l'azote liquide doit être non inflammable.
  \item 
\end{itemize}

\subsection*{Pression}
\paragraph*{Dangers}
L'azote liquide s'évapore très rapidement et peut entraîner de grandes augmentations de volume. Ceci peut se traduire par des explosions.
\paragraph*{Mesures de sécurité}
Le transport se fait en utilisant un dewar avec un couvercle perméable qui permette l'échappement du gaz. Le récipient principal doit contenir des valves permettant l'échappement du gaz lors de surpressions.

\subsection*{Hypoxie}
\paragraph*{Dangers}
Point relié au précédent, lors de l'évaporation de l'azote liquide peut déplacer l'air contenu dans la salle et provoquer un manque d'oxygène appelé hypoxie. Lorsque le taux d'oxygène dans l'air passe en dessous de 17\% il y a un risque d'évanouissement. De plus le corps n'arrive pas à détecter le manque d'oxygène et il est donc très difficile de réagir avant l'évanouissement.
\paragraph*{Mesures de sécurité}
Il faut travailler dans une zone aérée, surtout dans la salle où se trouve le récipient principal. Il ne faut pas manipuler l'azote liquide seul.




