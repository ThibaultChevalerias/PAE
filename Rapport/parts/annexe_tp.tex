Disposer de deux matériaux semiconducteurs, un intrinsèque et un dopé.

Objectifs :
\begin{enumerate}
\item trouver lequel est le dopé, lequel est intrinsèque ;
\item identifier s'il le semiconducteur dopé est de type N ou P.
\end{enumerate}

\bigskip
Première manipulation pour la discrimination des échantillons :

En utilisant une méthode de mesure de conductivité \textbf{4 pointes}, réaliser une mesure de conductivité de 
chaque échantillon en fonction de la température.

On utilisera pour la mesure de température une sonde Pt100 de gamme [$\unit{-200}\celsius$ ; $\unit{+800}\celsius].
Le porte échantillon sera remplie de billes de verre dans lesquelles on enfouira la sonde. On posera l'échantillon sur
le tapis de billes de verre en contact avec la sonde.

On refroidira le tout à l'azote liquide, en prenant soin que la thermalization se fasse avant de prendre les mesures.

\bigskip
Deuxième manipulation pour la détermination du type du semiconducteur dopé :

Placer l'échantillon dopé dans l'entrefer d'un électro-aimant.

Mesurer le champ magnétique $B$ (à l'aide d'un SQUID) et la différence de potentiel aux bornes de l'échantillon.

Déduire du signe de la tension Hall la nature des porteurs libres, et donc le type du semiconducteur.



