Disposer de deux matériaux semi-conducteurs, un intrinsèque et un dopé.

Objectifs :
\begin{enumerate}
\item trouver lequel est dopé, lequel est intrinsèque ;
<<<<<<< HEAD
\item identifier si le semi-conducteur dopé est de type N ou P.
=======
\item identifier si le semiconducteur dopé est de type N ou P.
>>>>>>> origin/master
\end{enumerate}

\bigskip
Première manipulation pour la discrimination des échantillons :

En utilisant une méthode de mesure de conductivité \textbf{4 pointes}, réaliser une mesure de conductivité de 
chaque échantillon en fonction de la température.

On utilisera pour la mesure de température une sonde Pt100 de gamme [$\unit{-200}\celsius$ ; $\unit{+800}\celsius$].
Le porte-échantillon sera rempli de billes de verre dans lesquelles on enfouira la sonde. On posera l'échantillon sur
le tapis de billes de verre en assurant un contact avec la sonde.

On refroidira le tout à l'azote liquide, en prenant bien soin que la thermalisation se fasse avant de prendre les mesures.

\bigskip
Deuxième manipulation pour la détermination du type du semi-conducteur dopé :

Placer l'échantillon dopé dans l'entrefer d'un électro-aimant.

Mesurer le champ magnétique $B$ (à l'aide d'un SQUID) et la différence de potentiel aux bornes de l'échantillon.

Déduire du signe de la tension Hall la nature des porteurs libres, et donc le type du semi-conducteur, en se basant sur la formule :

\begin{equation*}
V_{H} = \frac{IB}{enq}
\end{equation*}

où $\left\{
    \begin{array}{ll}
        I $ est l'intensité tranversant l'échantillon$\\
        B $ est le champ magnétique$\\
        e $ est l'épaisseur de l'échantillon$\\
        n $ est la densité de porteurs$\\
        q $ est la charge des porteurs$
    \end{array}
\right.$

\paragraph{Remarque :}
Si le temps le permet, calculer la densité des porteurs $n$.
