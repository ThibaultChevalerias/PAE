\bigskip
On a du réaliser quelques commandes pour avoir tout le matériel nécessaire pour réaliser notre éxpérience.

\begin{itemize}
  \item Les deux wafers de silicium à 20 euros l'unité plus 80 euros de coût d'envoi.
  \item Un câble \emph{smart488} à 190 plus un coût d'envoi.
\end{itemize}

\subsection{Marque, modèle, fournisseur(s)}
Les échantillons de silicium ont été commandés à travers le site www.universitywafer.com . Nous avons commandé un wafer de silicium dopé plus, de référence .... et un dopé moins de référence ....
Le câble \emph{smart488} est fabriqué par la compagnie \emph{Alciom} et on l'a commandé au fournisseur .....

Les dépenses réalisées pour notre éxpérience s'élèvent donc uniquement à 310 euros, ce qui est un prix considérable pour cette expérience. Cependant les deux câbles \emph{smart488} sont utilisés sont utilisés pour d'autres éxpériences telles que la supraconductivité et vont sûrement être réutilisés pendant plusieurs années tout comme les échantillons de silicium. Il ne devrait donc pas y avoir beaucoup de dépenses à faire pour les prochaines séances d'AE.
Il est possible de réaliser deux commandes supplémentaires pour compléter le matériel de la manipulation.
Tout d'abord et le plus important serait de commander des pointes de contact à ressort. Il est possible d'en acheter en ligne sur \emph{Castorama} au prix de huit euros cinquante la paire où dix-sept euros les quatres pointes plus le coût d'envoi.
Une deuxième commande plus optionelle correspondrait à l'achat d'un troisième échantillon de silicium non dopé pour ainsi comparer sa caractéristique avec les échantillons dopés et compléter la gamme de semiconducteurs. Cet échantillon est aussi disponible sur \emph{UniversityWafer} au prix unitaire de soixante quinze-euros. Le prix est plus élevé que les échantillons dopés et nous voulions pas surcharger le budget avant d'avoir quelques résultats positifs mais au vu de ce que l'on a obtenu il serait envisageable de passer cette commande.
