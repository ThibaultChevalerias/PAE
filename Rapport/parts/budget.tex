\bigskip
On a du réaliser quelques commandes pour avoir tout le matériel nécessaire pour réaliser notre éxpérience.

\begin{itemize}
  \item Les deux wafers de silicium à 20 euros l'unité plus 80 euros de coût d'envoi.
  \item Un câble \emph{smart488} à 190 plus un coût d'envoi.
\end{itemize}

\subsection{Marque, modèle, fournisseur(s)}
Les échantillons de silicium ont été commandés à travers le site www.universitywafer.com . Nous avons commandé un wafer de silicium dopé plus, de référence .... et un dopé moins de référence ....
Le câble \emph{smart488} est fabriqué par la compagnie \emph{Alciom} et on l'a commandé au fournisseur .....

Les dépenses réalisées pour notre éxpérience s'élèvent donc uniquement à 310 euros, ce qui est un prix considérable pour cette expérience. Cependant les deux câbles \emph{smart488} sont utilisés sont utilisés pour d'autres éxpériences telles que la supraconductivité et vont sûrement être réutilisés pendant plusieurs années tout comme les échantillons de silicium. Il ne devrait donc pas y avoir de dépenses à faire pour les prochaines séances d'AE.
La seule dépense supplémentaire qui pourrait être utile correspondrait à l'achat d'un troisième échantillon de silicium non dopé. On ne voulait pas surcharger le budget avant d'avoir quelques résultats positifs mais au vu de ce que l'on a obtenu il serait envisageable de passer cette commande qui ne devrait pas coûter plus de quarante euros plus le prix d'envoi.
