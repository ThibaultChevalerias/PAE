L'éxpérience que l'on a mise en place fait intervenir des équipements assez coûteux mais qui sont utiles pour une vaste gamme d'éxprériences différentes. Elle demande aussi un minimum de matériél assez particulier. Voici tout d'abord tous les équipements que nous avons utilisés, avec leur prix estimé:
\begin{itemize}
  \item Deux multimètres flukes 8840A , 400 euros l'unité.
  \item Deux sondes Pt100 à un prix de 27 euros l'unité, plus le coût d'envoi.
  \item Une licence Labview au prix de 1200 euros la version standard.
  \item Une plaque chauffante, 200 euros.
  \item Deux wafers de silicium dopés, 20 euros l'unité.
  \item Une plaque de cuivre ~ 5 euros.
  \item Billes de verre ~ 10 euros.
  \item Une potence ~ 10 euros. 
  \item Des cables ~ 10 euros.
  \item Gants isolants pour manipuler l'azote, 60 euros.
  \item Perceuse, 100 euros.
\end{itemize}
Il y a aussi l'utilisation de l'azote liquide pour lequel le prix est difficile à estimer. 
Tout cela nous donne un total qui tourne autour de 2500 euros, sans compter l'azote ni le coût d'envoi du matériel.
C'est un coût considérable pour une éxpérience simple, mais la plupart du matériel utilisé est utilisé par le laboratoire pour de nombreuses autres manipulations. 
Si on regarde plutôt le matériel utilisé uniquement pour notre éxpérience nous avons:
\begin{itemize}
  \item Les deux wafers de silicium à 20 euros l'unité plus 80 euros de coût d'envoi.
  \item La plaque de cuivre
\end{itemize}
La plaque de cuivre était disponible au laboratoire, nous avons dû commander les wafers. Les dépenses s'élèvent donc uniquement a 120 euros, ce qui est un prix raisonnable pour cette expérience. De plus le silicium peut être réutilisé plusieurs années de suite. 
Une dépense supplémentaire qui pourrait être utile correspondrait à l'achat d'un troisième échantillon de silicium non dopé. On ne voulait pas surcharger le budget avant d'avoir quelques résultats positifs mais au vu de ce que l'on a obtenu il est envisageable de réaliser cette commande.
