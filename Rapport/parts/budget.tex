L'expérience que l'on a mis en place fait intervenir des équipements assez coûteux mais qui sont utiles pour une vaste gamme d'expérimentations différentes. Elle demande aussi un minimum de matériel très particulier.

\bigskip
Voici tout d'abord tous les équipements que nous avons utilisés, avec leur prix estimé:

\begin{itemize}
  \item Deux multimètres \emph{Fluke 8840A} , 400 euros l'unité.
  \item Deux sondes \emph{Pt100} à un prix de 27 euros l'unité, plus le coût d'envoi.
  \item Une licence \emph{Labview} au prix de 1200 euros la version standard.
  \item Une plaque chauffante, 200 euros.
  \item Deux wafers de silicium dopés, 20 euros l'unité.
  \item Une plaque de cuivre, environ 5 euros.
  \item Un sachet de billes de verre, environ 10 euros.
  \item Une potence, environ 10 euros. 
  \item Des câbles, environ 10 euros.
  \item Des gants isolants pour manipuler l'azote liquide, 60 euros.
  \item Une perceuse, pour réaliser le support pour nos sondes (température et mesure de résistance), 100 euros.
\end{itemize}

\bigskip
Il y a aussi l'utilisation de l'azote liquide pour lequel le prix est difficile à estimer. 
Tout cela nous donne un total d'environ 2500 euros, sans compter l'azote ni le coût d'envoi du matériel.
C'est un coût considérable pour une expérience simple comme celle-ci, mais la plupart du matériel utilisé est partagé au sein du laboratoire pour de nombreuses autres manipulations.

\bigskip
Si on regarde plutôt le matériel utilisé uniquement pour notre expérience nous avons:

\begin{itemize}
  \item Les deux wafers de silicium à 20 euros l'unité plus 80 euros de coût d'envoi.
  \item La plaque de cuivre
\end{itemize}

La plaque de cuivre était disponible au laboratoire, nous avons dû commander les wafers. Les dépenses s'élèvent donc uniquement a 120 euros, ce qui est un prix raisonnable pour cette expérience. De plus le silicium peut être réutilisé plusieurs années de suite. 

\bigskip
Une dépense supplémentaire qui pourrait être utile correspondrait à l'achat d'un troisième échantillon de silicium non dopé. On ne voulait pas surcharger le budget avant d'avoir quelques résultats positifs mais au vu de ce que l'on a obtenu il serait envisageable de réaliser cette commande.
