Pour effectuer l'étude du matériau considéré, on mesure la résistance de l'échantillon grâce à une mesure 4 pointes.
On mesure en même temps la résistance d'un sonde de température qui a été étalonnée.
Donc, à partir de ces données on peut en déduire la conductivité en fonction de la température.

\paragraph{Pourquoi une mesure 4 pointes (et pas tout simplement 2) ?}
La mesure 4 pointes consiste à injecter un courant avec deux pointes et de mesurer une tension avec deux autres placées entre les deux premières.
Cette méthode permet de s'affranchir des résistances des fils et des résistances de contact.
Elle est donc particulièrement utile pour mesurer des résistances de l'ordre de l'$\Omega$ ou inférieures, car c'est l'ordre de grandeur des résistance parasites.
Par contre, elle est parfaitement inutile pour les hautes résistances (plusieurs k$\Omega$, M$\Omega$), donc on aurait mesuré en 2 pointes pour un isolant par exemple.

\paragraph{Principe de la sonde, comment a-t-on étalonné la sonde ?}
Nous avons étalonné la sonde Pt en deux étapes :
    - en la plaçant dans de l'eau que nous avons fait bouillir, nous avons pu mesurer la résistance de la sonde à une température de 100°C.
    _ en la plaçant ensuite dans de l'eau pleine de glaçons, nous avons fait la même chose pour 0°C.
La résistance du platine augmente linéairement avec la température, c'est une propriété de ce matériau.
On obtient alors la caratéristique R(T) : en mesurant la résistance du Platine, on peut en déduire sa température (donc idéalement la température de l'échantillon).

\paragraph{Interfaçage grâce à Labview}
Nous avons utilisé Labview pour réaliser l'acquisition des mesures de nos deux multimètres de façon synchronisée, et écrire les résultats dans un fichier lisible par Excel (ou gnuplot).
Nous l'avons utilisé pour la mesure sur le Cuivre, mais nous l'avons ensuite abandonné pour la mesure sur le Silicium pour éliminer les sources potentielles de problèmes (on a simplifié le montage au maximum).
[Parler du code et des avantages de Labview]
