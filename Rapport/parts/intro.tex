Le phénomène physique à traiter dans le cadre de cette activité expérimentale de physique était la « conduction ». 
Nous étions le premier groupe à travailler sur ce sujet et démarrions de rien. 
Nous avons donc choisi de nous intéressé à la conductivité \textit{électrique}.

La conductivité électrique représente la capacité d'un corps à conduire le \textit{courant électrique} ; 
ce qu'on entend par « courant électrique », c'est un déplacement de charges électriques 
(ces charges peuvent être positives ; nous y reviendrons).

\bigskip
Le fil conducteur de notre démarche fut l'idée d'AE à proposer aux élèves de première année, 
idée qui se base sur le constat suivant :
\textbf{On peut caractériser incontestablement un semiconducteur (SC) en effectuant 
une mesure de conductivité en fonction de la température.}
Par caractériser on entend déterminer s'il s'agit d'un SC pur, dopé voire dégénéré.

\bigskip
L'idée d'AE est donc la suivante :
Mettre à disposition des élèves de première année 2 types de SC, un pur et un dopé. 
Objectif : déterminer qui est qui, en réalisant une étude de la conductivité de chaque échantillon 
en fonction de la température.

Une remarque à ce stade : cette étude de conductivité ne permet pas de dire si le SC dopé est de type N ou P.





La manipulation que nous avons mise en place consiste à mesurer la conductivité d'un matériau 
en fonction de sa température. Nous nous sommes principalement intéressés au Silicium, 
mais nous avons également étudié le Cuivre et il serait aussi possible de réaliser 
l'expérience avec un autre matériau, par exemple un isolant.
