Le présent rapport détaille le projet d'activité expérimentale (AE) s'inscrivant dans le cadre de l'option de troisième année 
\og Physique et Applications \fg{} de l'École Centrale Paris, et s'étant déroulé du 9 septembre au 30 octobre 2014.

\bigskip
Notre trinôme, constitué de Thibault \bsc{Chevalérias}, Emmanuel \bsc{Lassalle} et Ilan \bsc{Shlesinger}, était encadré 
par Charles \bsc{Paillard}, en deuxième année de thèse au Laboratoire SPMS\footnote{Laboratoire Structures, 
Propriétés et Modélisation des Solides} de l'École Centrale Paris.

\bigskip
Parmi les sujets proposés, qui consistaient en l'étude d'un phénomène physique particulier, nous avons choisi le thème 
\og mesures de conductivités \fg. En effet, la microélectronique occupant désormais une importance vitale dans notre société, 
la compréhension des mécanismes de conduction électrique devient cruciale notamment dans 
la recherche de nouveaux matériaux pour les nouvelles technologies de l'information.

\bigskip
L'esprit de ce projet d'AE était orienté vers une formation à la démarche scientifique plus que vers l'apprentissage
de techniques particulières. De plus, il s'agissait concrètement de mettre en place une AE susceptible d'être réalisée
par des élèves de première année.

\bigskip
Ce rapport présente la démarche scientifique suivie ainsi que la mise en \oe uvre concrète de la manipulation
imaginée, centrée autour du thème de la mesure de conductivités.

\bigskip
Par ailleurs, nous remercions Charles \bsc{Paillard} pour son soutien et son aide précieuse tout au long du projet, 
qui s'est révélé ne pas être un long fleuve tranquille mais plutôt un parcours du combattant, avec des obstacles à
franchir et de nombreuses contraintes à prendre en compte. 
Nous remercions également Pierre-Eymeric \bsc{Janolin} pour ses conseils au sujet de l'esprit du projet. 
Enfin, nous sommes particulièrement reconnaissants envers Gloria \bsc{Foulet}, Clarisse \bsc{Malbrun} et Franz 
\bsc{Wehling} pour leur assistance technique précieuse et efficace.

