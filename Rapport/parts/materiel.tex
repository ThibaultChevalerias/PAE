\subsection{Type de matériel et prix}
L'expérience que l'on a mise en place fait intervenir des équipements assez coûteux mais qui sont utiles pour une vaste gamme d'expérimentations différentes. Elle demande aussi un minimum de matériel très particulier.

\bigskip
Voici tout d'abord tous les équipements que nous avons utilisés, avec leur prix estimé:

\begin{itemize}
  \item Deux multimètres \emph{Fluke 8840A} de la compagnie \emph{Fluke} à 400 euros l'unité.
  \item Deux sondes \emph{Pt100} de la compagnie \emph{RS components} à un prix de 27 euros l'unité, plus le coût d'envoi.
  \item Une licence \emph{Labview} de la compagnie \emph{National Instruments} au prix de 1200 euros la version standard.
  \item Deux câbles \emph{smart488} d'\emph{Alciom} à un prix unitaire de 190 euros.
  \item Une plaque chauffante, 200 euros.
  \item Deux wafers de silicium dopés, 20 euros l'unité.
  \item Une plaque de cuivre, environ 5 euros.
  \item Un sachet de billes de verre, environ 10 euros.
  \item Une potence, environ 10 euros. 
  \item Des câbles conducteurs, environ 10 euros.
  \item Des gants isolants pour manipuler l'azote liquide, 60 euros.
  \item Une perceuse, pour réaliser le support pour nos sondes (température et mesure de résistance), 100 euros.
\end{itemize}

\bigskip
Il y a aussi l'utilisation de l'azote liquide pour lequel le prix est difficile à estimer. Le prix du litre est peu élevé, 
de l'ordre de cinq centimes d'euros par litre, mais il est nécéssaire d'avoir des réservoirs particuliers assez coûteux,
appelés « dewar ».
Tout cela nous donne un total d'environ 2900 euros, sans compter l'azote ni le coût d'envoi du matériel.
C'est un coût considérable pour une expérience simple comme celle-ci, mais la plupart du matériel était déjà présent au laboratoire et on l'utilise aussi pour de nombreuses autres manipulations.


\subsection{Manuels d'utilisation}
Les manuels d'utilisation sont en copie papier pour les câbles \emph{smart488} ou sur le site d'\emph{Alciom} www.alciom.com. Pour le \emph{Fluke 8840A} il est possible de le trouver sur le site du producteur www.fluke.com. Pour la documentation \emph{Labview}, se référer au site web de \emph{National Instruments}.

On va par la suite voir quels ont été les devis et commandes réalisés spécialement dans le cadre de notre activité.
