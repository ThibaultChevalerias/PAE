Cette activité a été parsemée de difficultées éxpérimentales qui nous ont ouvert les yeux sur les différences entre la théorie et l'éxpérience. Nous avons détaillé les points particulièrement problématiques dans la partie \emph{diificultés expérimentales}. Il est pourtant évident que nous n'avons pas traité ces problèmes de façon idéale et avec le recul nous avons pu établir quelques points qui serait à refaire différemment.


Comme premier point à améliorer se trouve le manque d'organisation lors de la recherche de problème. Il y avait différents paramètres qui pouvaient avoir une forte influence sur la mesure comme par exemple l'outil informatique, les instruments de mesures, le câblage ou encore les sondes de températures. C'est donc un manque de rigueur et de méthodologie dans la recherche du facteur problématique qui nous a été le plus pénalisant niveau temps. En effet après avoir eu de bons résultats avec le cuivre nous avons voulu enchaîner directement sur la mesure la plus intéressante, et donc la plus compliquée, dès le début. C'est la mesure à basse température sur l'échantillon semiconducteur. Cependant les valeurs obtenues n'étaient pas du tout cohérentes, et cela était dû à divers problèmes dans le montage. Il aurait donc fallu ajouter des paramètres un à un pour vérifier le bon fonctionnement de ceux-ci, c'est à dire réaliser la mesure à température ambiante puis utiliser l'azote liquide et enfin utiliser la prise de donnée informatique. 

Un deuxième point à améliorer concerne aussi la méthode de travail et plus particulièrement l'intégration de labview pour réaliser les mesures. Nous avions tout trois peu d'expérience dans l'automatisation des mesures et nous avons sous-estimé la difficulté que représente le travail pour obtenir les données sur ordinateur. Nous y avons donc dépensé la plus grande partie de notre temps au dépens des mesures réelles. Nous avons finalement obtenu un programme qui fonctionne et une meilleure connaissance de la connectique reliée à Labview mais il nous a manqué du temps pour réaliser plus de mesures. Donc si cela était à refaire il faudrait mieux s'organiser pour ne pas bloquer sur l'interfaçage mais plutôt continuer en parallèle les éxpériences manuelles et ainsi avoir plus de temps pour résoudre les problèmes qui proviennent de cette partie. Il faudrait cependant réaliser un bon débriefing à plusieurs reprises dans la journée pour se maintenir au courant des diverses progressions de chacun. 

Un troisième point bien plus simple cette fois concerne le contact avec nos échantillons. Comme précisé dans la partie expérimentale nous avons eu du mal à réaliser un bon contact, ceci surtout dû à une pression variable des pointes conductrices. Nous avons identifié ce problème le dernier jour à cause de la perte de temps détaillée dans le paragraphe précédent.  Un moyen d'améliorer ce point serait donc d'utiliser des pointes de test à ressorts.
