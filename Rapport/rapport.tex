\documentclass{article}


% Liste des packages qu'on va utiliser
\usepackage[utf8]{inputenc} 
\usepackage[T1]{fontenc}      
\usepackage[francais]{babel}

\usepackage{amsmath}
\usepackage{amssymb}
\usepackage{mathrsfs}
\usepackage[squaren, Gray, cdot]{SIunits} %unités

\usepackage{setspace}             % changer l'interlignage
\newcommand{\HRule}{\rule{\linewidth}{0.5mm}}

\usepackage{graphicx}

%Définition des couleurs
\usepackage[usenames]{color}
	\definecolor{Orange}{rgb}{0.9,0.5,0.1}
	\definecolor{Vert}{rgb}{0.2,0.55,0.3}

% Début du document
\begin{document}

% Inspiré de http://en.wikibooks.org/wiki/LaTeX/Title_Creation

\begin{titlepage}

\begin{center}

\begin{minipage}[t]{0.48\textwidth}
  \begin{flushleft}
    \includegraphics [width=30mm]{figures/ecp.jpg} \\[0.5cm]
    \begin{spacing}{1.5}
      \textsc{\'Ecole Centrale Paris}
    \end{spacing}
  \end{flushleft}
\end{minipage}
\begin{minipage}[t]{0.48\textwidth}
  \begin{flushright}
    \includegraphics [width=30mm]{figures/spms.jpg} \\[0.5cm]
    \textsc{Laboratoire Structures, Propriétés et Modélisation
      des Solides}
  \end{flushright}
\end{minipage} \\[3cm]

\textsc{\Large Atelier Expérimental --- 3\ieme{} année \\Option Physique
et Applications}\\[0.5cm]
\HRule \\[0.4cm]
{\huge \bfseries MESURES DE CONDUCTIVIT\'ES}\\[0.4cm]
\HRule \\[1.3cm]

\begin{minipage}[t]{0.3\textwidth}
  \begin{flushleft} 
    \emph{\'Elèves :}\\
    Thibault \textsc{Chevalérias}\\
    Emmanuel \textsc{Lassalle}\\
    Ilan \textsc{Shlesinger}\\
  \end{flushleft}
\end{minipage}
\begin{minipage}[t]{0.6\textwidth}
  \begin{flushright} 
    \emph{Encadrant :} \\
    Charles \textsc{Paillard} \\(doctorant)\\
  \end{flushright}
\end{minipage}

\vfill

{\large Du 09 septembre au 30 octobre 2014}

\end{center}

\end{titlepage}


\tableofcontents

\newpage

\section{{Introduction}}
[Manu, tu peux écrire pas mal de ce que t'avais préparé pour la soutenance ici je pense]

La manipulation que nous avons mise en place consiste à mesurer la conductivité d'un matériau en fonction de sa température. Nous nous sommes principalement intéressés au Silicium, mais nous avons également étudié le Cuivre et il serait aussi possible de réaliser l'expérience avec un autre matériau, par exemple un isolant.


\section{{Objectifs}}
\input{./parts/objectifs.tex}

\section{{Matériel}}
Penser aux 2 flukes, aux sondes (dont celle qu'on a cassée ^^), à la licence Labview, aux wafers, à la plaque de Cuivre, aux ordis, aux billes de verre, aux supports en polystyrène, aux portes-matériels, à l'azote liquide, à la plaque chauffante, et j'en oublie probablement.


\section{{Budget}}
Penser aux 2 flukes, aux sondes (dont celle qu'on a cassée ^^), à la licence Labview ? (je pense pas), aux wafers, à la plaque de Cuivre, aux ordis ? (je pense pas), aux billes de verre, aux supports ? (c'est du polystyrène qui coûte rien a priori), aux portes-matériels, à l'azote liquide, à la plaque chauffante, et j'en oublie probablement.


\section{{Protocoles expérimentaux}}
\section*{Montages}
Pour effectuer l'étude du matériau considéré, on mesure la résistance de l'échantillon grâce à une mesure 4 pointes.
On mesure en même temps la résistance d'un sonde de température qui a été étalonnée.
Donc, à partir de ces données on peut en déduire la conductivité en fonction de la température.


Nous avons commencé avec une mesure sur un conducteur pour partir sur quelque chose de simple.
La tendance attendue pour le Cuivre semblait être facile à obtenir, donc nous avons débuté avec ce métal.


Puis nous avons voulu effectué une mesure sur un semi-conducteur, nous avons donc commandé deuw wafers de Silicium (dopés n et p), et poursuivi les mesures avec ce Silicium.


\subsection*{Schémas}

\subsection*{Photos}


\section*{Conditions expérimentales}
Nous avons utilisé Labview pour réaliser les acquisitions des deux résistances mesurées.
Nous avons simplement posé la plaque de Cuivre sur une plaque chauffante et mesuré sa résistance en fonction de la température en 4 pointes, entre 25°C et 205°C.


Pour le Silicium, nous voulions détecter un palier de conductivité, et nous savions que ce palier risquait de se trouver à une température inférieure ou égale à la température ambiante.
Donc, pour pouvoir étudier ce palier, nous avons dû baisser fortement la température et nous avons utilisé de l'azote liquide pour cela.
Nous avons également placé des billes de verre dans notre porte-échantillon pour augmenter l'inertie thermique du montage et nous permettre d'avoir une montée en température relativement lente.
La sonde de température a été insérée dans le porte-échantillon par un trou percé sur le côté (voir montage). De cette façon, elle peut être en contact du Silicium et être entourée par les billes de verre.
De ce fait, elle mesure bien mieux la température du silicium que si on l'avait simplement posée dessus.


\paragraph*{Pourquoi une mesure 4 pointes (et pas tout simplement 2) ?}
La mesure 4 pointes consiste à injecter un courant avec deux pointes et de mesurer une tension avec deux autres placées entre les deux premières.
Cette méthode permet de s'affranchir des résistances des fils et des résistances de contact.
Elle est donc particulièrement utile pour mesurer des résistances de l'ordre de l'$\Omega$ ou inférieures, car c'est l'ordre de grandeur des résistance parasites.
Par contre, elle est parfaitement inutile pour les hautes résistances (plusieurs k$\Omega$, M$\Omega$), donc on aurait mesuré en 2 pointes pour un isolant par exemple.


\paragraph*{Principe de la sonde, comment a-t-on étalonné la sonde ?}
Nous avons étalonné la sonde Pt en deux étapes :


\begin{itemize}
  \item en la plaçant dans de l'eau que nous avons fait bouillir, nous avons pu mesurer la résistance de la sonde à une température de 100°C.
  \item en la plaçant ensuite dans de l'eau pleine de glaçons, nous avons fait la même chose pour 0°C.
\end{itemize}


La résistance du platine augmente linéairement avec la température, c'est une propriété de ce matériau.
On obtient alors la caratéristique R(T) : en mesurant la résistance du Platine, on peut en déduire sa température (donc idéalement la température de l'échantillon).


\paragraph*{Interfaçage grâce à Labview}
Nous avons utilisé Labview pour réaliser l'acquisition des mesures de nos deux multimètres de façon synchronisée, et écrire les résultats dans un fichier lisible par Excel (ou gnuplot).
Nous l'avons utilisé pour la mesure sur le Cuivre, mais nous l'avons ensuite abandonné pour la mesure sur le Silicium pour éliminer les sources potentielles de problèmes (on a simplifié le montage au maximum).
[Parler du code et des avantages de Labview]


\section*{Difficultés expérimentales}


\section*{Résultats}
Pour le Cuivre, nous avons simplement observé que la conductivité diminue quand on augmente la température (voir courbe).
[donner un ordre de grandeur de la conductivité du Cuivre sur cette plage de température si elle est bonne]


Pour le Silicium, nous avons observé une conductivité croissante avec la température, ce qui est cohérent. Nous avons aussi observé un palier aux alentours de [20°C; 80°C], ce qui semble cohérent aussi.
À haute température, les contacts se sont dégradés d'où des points un peu épars...


\subsection*{Tableaux}

\subsection*{Courbes}


\section*{Discussion des résultats}
L'endroit où se trouve le palier paraît cohérent, pour le Si il es autour de Tamb, c'est pourquoi le Si est utilisé dans l'électronique.


[Tu pourrais préciser ça Manu ? (c'est ce qu'on avait noté pour l'oral) : 
Pour le fait qu'on identifie qu'il s'agit d'un échantillon dopé je parle des différentes allures de conductivités en fonction de T dans ma partie, on se raccrochera à ça.
Pour la détermination du dopage, il faut repérer la conductivité de notre échantillon de Si à Tamb et à l'aide d'abaques en déduire le dopage (j'ai vite fait essayé, sans valeurs de conductivités précises, et je trouve un dopage de 10^20 porteurs par cm^3)
    [double plateau, essaye de lire sur wiki pour bien comprendre on risque d'avoir des questions dessus/ le type de dopage n'est pas visible directement, peut etre en faisant un truc special avec la vitesse des porteurs mais je ne crois pas]]


\section*{Conclusions}



\section{{Retour d'expérience et perspectives}}
Cette activité a été parsemée de difficultés expérimentales qui nous ont ouvert les yeux sur les différences entre la théorie et l'expérience. Nous avons détaillé les points particulièrement problématiques dans la partie \emph{difficultés expérimentales}. Il est pourtant évident que nous n'avons pas traité ces problèmes de façon idéale et avec le recul nous avons pu établir quelques points qui serait à refaire différemment.

Malgré les difficultés liées au montage expérimental, les manipulations ont assez bien fonctionné et les résultats obtenus sont cohérents.
Avec plus de temps, il aurait sans doute été possible de réaliser plus de mesures, avec différents types de semi-conducteurs, différents types de dopage, ou encore des isolants.
Bien entendu, nous aurions également souhaité effectuer le montage Hall pour pouvoir compléter la mesure de conductivité.
Aussi, nous aurions aussi souhaité raffiner notre code Labview et mieux interfacer la mesure avec celui-ci, rendant ainsi la manipulation plus aisée.

Comme premier point à améliorer se trouve le manque d'organisation lors de la recherche de problème. Il y avait différents paramètres qui pouvaient avoir une forte influence sur la mesure comme par exemple l'outil informatique, les instruments de mesures, le câblage ou encore les sondes de températures. C'est donc un manque de rigueur et de méthodologie dans la recherche du facteur problématique qui nous a été le plus pénalisant niveau temps. En effet après avoir eu de bons résultats avec le cuivre nous avons voulu enchaîner directement sur la mesure la plus intéressante, et donc la plus compliquée, dès le début. C'est la mesure à basse température sur l'échantillon semiconducteur. Cependant les valeurs obtenues n'étaient pas du tout cohérentes, et cela était dû à divers problèmes dans le montage. Il aurait donc fallu ajouter des paramètres un à un pour vérifier le bon fonctionnement de ceux-ci, c'est à dire réaliser la mesure à température ambiante puis utiliser l'azote liquide et enfin utiliser la prise de donnée informatique. 

Un deuxième point à améliorer concerne aussi la méthode de travail et plus particulièrement l'intégration de labview pour réaliser les mesures. Nous avions tout trois peu d'expérience dans l'automatisation des mesures et nous avons sous-estimé la difficulté que représente le travail pour obtenir les données sur ordinateur. Nous y avons donc dépensé la plus grande partie de notre temps au dépens des mesures réelles. Nous avons finalement obtenu un programme qui fonctionne et une meilleure connaissance de la connectique reliée à Labview mais il nous a manqué du temps pour réaliser plus de mesures. Donc si cela était à refaire il faudrait mieux s'organiser pour ne pas bloquer sur l'interfaçage mais plutôt continuer en parallèle les éxpériences manuelles et ainsi avoir plus de temps pour résoudre les problèmes qui proviennent de cette partie. Il faudrait cependant réaliser un bon débriefing à plusieurs reprises dans la journée pour se maintenir au courant des diverses progressions de chacun. 

Un troisième point bien plus simple cette fois concerne le contact avec nos échantillons. Comme précisé dans la partie expérimentale nous avons eu du mal à réaliser un bon contact, ceci surtout dû à une pression variable des pointes conductrices. Nous avons identifié ce problème le dernier jour à cause de la perte de temps détaillée dans le paragraphe précédent.  Un moyen d'améliorer ce point serait donc d'utiliser des pointes de test à ressorts.


\newpage

\appendix

\section{{Sécurité}}
La sécurité est un sujet primordial dans tous les laboratoires et tout travail comportant des risques se doit d'être encadré par des règles strictes. Notre activité expérimentale comportait différentes manipulation à risque que nous avons identifiées:

\begin{itemize}
  \item Manipulation de l'azote liquide.
  \item Manipulation de la plaque chauffante.
  \item Manipulation d'instruments à haute tension.
  \item Manipulation de la perçeuse.
\end{itemize}

\bigskip
Pour chacun de ces aspects nous avons pris les précautions nécessaires pour minimiser les risques d'accidents. 
L'azote liquide est utilisé dans divers domaines de l'industrie et de la recherche. Nous l'avons utilisé pour pouvoir refroidir notre échantillon à très basse température. En effet la température de l’azote liquide (point d’ébullition à la pression atmosphérique) est de -196\celsius{}. 
Il existe quatre catégories principales de dangers reliés à la manipulation de l'azote liquide. Nous allons les présenter individuellement suivies des mesures de sécurité à respecter.

\subsection{Très basse température}

\paragraph{Dangers}

\begin{itemize}
  \item Lorsque le liquide cryogénique entre en contact avec la peau, il peut provoquer des 
brûlures par le froid appelées gélures. Des gélures sur une grande surface de la peau peuvent être mortelles. 
  \item Les très basses températures diminuent la résilience et la ductilité de certains 
matériaux, qui se fragilisent et peuvent se briser facilement. Ceci peut se traduire par des coupures ou par un renversement d'un produit dangereux.
  \item Lors de la manipulation de l'azote liquide l’air peut se condenser. Le condensat riche en oxygène augmente le risque d'incendie 
\end{itemize}

\paragraph{Mesures de sécurité}

\begin{itemize}
  \item Lors de toute manipulation de l'azote liquide il faut utiliser des gants approriés, des vêtements couvrant tout le corps, des lunettes de sécurité et des chaussures fermées.
  \item Le transport du liquide doit être réalisé dans des récipients adaptés, résistants aux basses températures et aux chocs. Lors du déplacement du récipient il faut éviter tout choc ou mouvement risquant le renversement du liquide. Le sol des pièces contenant l'azote liquide doit être non inflammable.
  \item 
\end{itemize}

\subsection{Pression}

\paragraph{Dangers}

L'azote liquide s'évapore très rapidement et peut entraîner de grandes augmentations de volume. Ceci peut se traduire par des explosions.
\paragraph{Mesures de sécurité}

Le transport se fait en utilisant un dewar avec un couvercle perméable qui permette l'échappement du gaz. Le récipient principal doit contenir des valves permettant l'échappement du gaz lors de surpressions.

\subsection{Hypoxie}

\paragraph{Dangers}

Point relié au précédent, lors de l'évaporation de l'azote liquide peut déplacer l'air contenu dans la salle et provoquer un manque d'oxygène appelé hypoxie. Lorsque le taux d'oxygène dans l'air passe en dessous de 17\% il y a un risque d'évanouissement. De plus le corps n'arrive pas à détecter le manque d'oxygène et il est donc très difficile de réagir avant l'évanouissement.
\paragraph{Mesures de sécurité}

Il faut travailler dans une zone aérée, surtout dans la salle où se trouve le récipient principal. Il ne faut pas manipuler l'azote liquide seul.






\section{Plan du TP}
Disposer de deux matériaux semi-conducteurs, un intrinsèque et un dopé.

Objectifs :
\begin{enumerate}
\item trouver lequel est dopé, lequel est intrinsèque ;
\item identifier si le semi-conducteur dopé est de type N ou P.
\end{enumerate}

\bigskip
Première manipulation pour la discrimination des échantillons :

En utilisant une méthode de mesure de conductivité \textbf{4 pointes}, réaliser une mesure de conductivité de 
chaque échantillon en fonction de la température.

On utilisera pour la mesure de température une sonde Pt100 de gamme [$\unit{-200}\celsius$ ; $\unit{+800}\celsius$].
Le porte-échantillon sera rempli de billes de verre dans lesquelles on enfouira la sonde. On posera l'échantillon sur
le tapis de billes de verre en assurant un contact avec la sonde.

On refroidira le tout à l'azote liquide, en prenant bien soin que la thermalisation se fasse avant de prendre les mesures.

\bigskip
Deuxième manipulation pour la détermination du type du semi-conducteur dopé :

Placer l'échantillon dopé dans l'entrefer d'un électro-aimant.

Mesurer le champ magnétique $B$ (à l'aide d'un SQUID) et la différence de potentiel aux bornes de l'échantillon.

Déduire du signe de la tension Hall la nature des porteurs libres, et donc le type du semi-conducteur, en se basant sur la formule :

\begin{equation*}
V_{H} = \frac{IB}{enq}
\end{equation*}

où $\left\{
    \begin{array}{ll}
        I $ est l'intensité tranversant l'échantillon$\\
        B $ est le champ magnétique$\\
        e $ est l'épaisseur de l'échantillon$\\
        n $ est la densité de porteurs$\\
        q $ est la charge des porteurs$
    \end{array}
\right.$

\paragraph{Remarque :}
Si le temps le permet, calculer la densité des porteurs $n$.


\section{Factures}
\label{annexe_factures}

\begin{figure}[hb]
  \begin{center}
		\includegraphics[height=6cm]{./images/facture.jpg}
		\caption{Facture pour le c\^able GPIB}
		\label{facture}
	\end{center}
\end{figure}

\newpage

\bibliographystyle{plain}
\bibliography{biblio}

\end{document}
